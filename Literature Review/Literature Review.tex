\documentclass{article}
\usepackage[utf8]{inputenc}
\usepackage{amsmath}
\usepackage{geometry}
\usepackage{natbib}
\geometry{a4paper, margin=1in}


\title{A Study on the Impact of Environmental Administrative Penalties on Corporate Green Technological Innovation}
\author{Yifan Mao}
\date{September 2025}
\begin{document}
\maketitle

\noindent \textbf{[JLG: Approved. You need to improve your filing naming, references, and typos. Some references are hardcoded and others are not.]}

\section*{1 Literature Review}

\subsection*{1.1 Research on Environmental Regulation}
Environmental regulation is the process of controlling and managing the environment through laws, policies, or other administrative means. These regulations aim to protect, maintain, or improve the natural environment to achieve harmony between humans and nature. According to the degree of government control, environmental regulations can be divided into three categories: command-and-control, market-based, and voluntary/persuasive (Song et al., 2018).

\subsubsection*{1) Command-and-Control Environmental Regulation}
Command-and-control environmental regulation is the dominant environmental management tool in China. Its advantages include clarity, equity, and ease of enforcement(Tan, 2018), and it has played a significant role in controlling pollution and improving environmental quality in the country. Based on the timing of control, this type of regulation can be divided into three categories: pre-control, in-process control, and post-control.

\paragraph{Pre-control} command-and-control regulations include systems like the "Environmental Impact Assessment (EIA)" and the "Three Simultaneities System." The EIA system requires that before a project begins construction, its potential environmental impacts from site selection, design, and future operation must be investigated and evaluated. Preventive measures must be conducted and submitted for legal approval (Wang, 2004). The "Three Simultaneities System" requires that environmental protection facilities must be designed and put into operation at the same time as the main project. These two pre-control measures reflect the policy goal of "preventing environmental pollution."

\paragraph{In-process control} command-and-control regulations include the "emission standards system" and "local government environmental targets." The emission standards system sets detailed technical standards for the concentration and total volume of pollutants that companies can discharge during production. Laws such as China's \textit{Environmental Protection Law}, \textit{Air Pollution Prevention and Control Law}, and \textit{Water Pollution Prevention and Control Law} all have clear requirements for corporate emissions, with standards varying across different industries. This system is the basis for China's environmental quality assessment and management. Raising pollution standards can push companies toward cleaner production and better pollution treatment, thereby reducing pollution levels (Zhang et al., 2023). To reverse the environmental degradation caused by China's extensive growth, the State Council made cadres' performance in environmental governance a criterion for their promotion in 2006. Local governments facing these targets have shown more significant progress in industrial upgrading (Yu et al., 2020), and these pollution controls have helped reduce resource misallocation in polluting industries and improve overall productivity (Han et al., 2020). However, the pressure of environmental targets also conflicts with short-term economic goals. Regions with high emission reduction targets have seen significant declines in their economic growth and fiscal revenue(Xie and Wang, 2022).

\paragraph{Post-control} command-and-control regulations include environmental administrative penalties and environment remediation liability systems. The rules from pre-control and in-process control provide the basis for imposing administrative penalties, while these penalties, in turn, ensure that companies and individuals comply with environmental regulations (Tan, 2018). Research on environment remediation liability mainly concentrates in the law field. This system provides an institutional guarantee for restoring ecosystems after damage, promoting harmony between humans and nature, and building a harmonious society (Li, 2013).

\subsubsection*{2) Market-Based Environmental Regulation}
Unlike command-and-control regulations, market-based regulations incentivize rather than force firms. They can be divided into Coasean and Pigouvian approaches, depending on whether they create a market or use the existing market.

\paragraph{Coasean approaches} Ronald Coase suggested that if property rights are well-defined and transaction costs are low enough, market transactions can lead to an optimal allocation of resources. Coasean tools mainly include carbon emissions trading, pollution discharge permit trading, and energy use rights trading. A carbon emissions trading policy first sets a maximum allowable level of carbon emissions for a region and then allocates emission permits, thereby controlling the total regional emissions. Companies that reduce their emissions can gain revenue by selling their surplus permits. This system can adjust the energy structure, improve regional technology levels, and thus achieve carbon reduction and green development (Dong and Wang, 2021). Pollution permit trading systems improve marketization and the development of factor markets by clearly defining property rights, which helps reduce regional energy consumption, promote green technology innovation, and improve green total-factor energy efficiency (Shi and Li, 2020). Energy use rights trading can control total energy consumption and offers significant economic and energy-saving potential for industries. However, this potential varies across industries, and in some sectors, energy-saving potential might be crowded out (Zhang and Zhang, 2019).

\paragraph{Pigouvian approaches} Arthur Pigou argued that market failures occur due to externalities. The government can correct these failures by imposing taxes or providing subsidies to align private costs with social costs, thereby maintaining market efficiency. Pigouvian tools mainly include the environmental protection tax and government subsidies. China's \textit{Environmental Protection Tax Law}, implemented on January 1, 2018, internalizes the external costs of pollution by setting tax standards for different pollutants, encouraging companies to reduce emissions and innovate. The environmental tax can improve the efficiency of fossil fuel use and incentivize green technology innovation aimed at pollution reduction (Liu and Xiao, 2022). A carbon tax might reduce output in the short term, but the impact is not substantial, and its long-term emission reduction effect is significant (Chen, 2011). There is an ongoing debate about whether government subsidies "stimulate" or "weaken" corporate innovation (Guo, 2018). The "stimulation" guys argue that subsidies can correct market failures in the innovation process and encourage companies to invest in R\&D (\citet{jaffe2015impact}). The "weakening" guys argue that information asymmetry between the government and firms can lead to over-investment (Wei et al., 2015). Furthermore, since environmental subsidies often target environmental construction projects rather than green technology innovation itself, opportunistic behavior by firms might "weaken" their motivation for green innovation (Li and Xiao, 2020).

\subsubsection*{3) Voluntary and Persuasive Regulation}
In a narrow sense, voluntary regulation refers to influencing environmental protection behavior through moral persuasion. In a broader sense, it includes all environmental regulation tools other than command-and-control and market-based types. Examples include environmental information disclosure and ESG (Environmental, Social, and Governance) ratings.

Environmental information disclosure can increase information symmetry, promote local green technology innovation and industrial structure upgrading, and thus achieve pollution and carbon reduction (Shao and Wang, 2024). Public disclosure of environmental information also affects a company's market value. After green rankings are published, the market value of top-ranked companies tends to rise, while that of bottom-ranked companies falls. The enforcement of laws like the \textit{Clean Air Act} has made the impact of environmental legal information on corporate market value even more obvious (Aaron et al., 2012; Badrinath and Bolster, 1996). ESG serves as a supplement to formal environmental regulation and is an important international standard for a company's green and sustainable development. Companies with good ESG performance typically have higher quality information disclosure and significantly lower financing costs (Qiu and Yin, 2019). Excellent ESG performance can also reduce a company's information and operational risks, leading to lower audit fees (Xiao et al., 2021). By easing financing constraints and improving employees' innovation and risk-taking, good ESG performance ultimately promotes corporate innovation (Fang and Hu, 2023).

To summarize, command-and-control regulation is highly deterministic and uses governmental authority to address environmental externalities. Given that the environment is a "public good," this type of regulation is essential. However, it can also lead to "government failure" due to insufficient information and a lack of flexibility, as it often applies a one-size-fits-all standard. Market-based regulation is more economically efficient and better at promoting continuous environmental improvement, but it faces challenges in pricing, market creation, and potential market failures. Voluntary regulation holds great promise and is highly flexible, but it relies on moral values and choices, which take a long time to foster.

\subsection*{1.2 Research on Green Technology Innovation}

In the early 1990s, academia began to focus on the negative environmental impacts of economic development and recognized the positive role of corporate innovation in improving the environment. However, there is still no consensus on the definition of green innovation. \citet{peter1997sustainability} defined green innovation as "new or modified processes, techniques, systems, and products to avoid or reduce environmental damage." Dong (2010), following the concept of the green economy, described green innovation as a complete process that includes innovation at the end-of-pipe, innovation in the production process, innovation in products, and innovation at the system level. \citet{driessen2013green} expanded the concept further, arguing that green innovation should not merely aim to reduce environmental burdens but should aim to produce significant environmental benefits. Current research often identifies corporate green technology innovation using the International Patent Classification (IPC) Green Inventory, published by the World Intellectual Property Organization (WIPO) in 2010 (Qi et al., 2018; Li and Xiao, 2020; Liu and Xiao, 2022). The WIPO Green Inventory covers seven main categories: alternative energy production, transportation, energy conservation, waste management, agriculture/forestry, administrative/regulatory/design aspects, and nuclear power generation. This provides a comprehensive way to identify green technology innovation. This paper adopts the WIPO classification to identify green technology innovation.

Green technology innovation can have a long-term impact on individual firms and the macroeconomy. Firms with more green technology innovations can produce more green products, differentiate themselves from competitors, and gain a green competitive advantage (\citet{barney1991firm}). \citet{clarke1994challenge} argued that when facing environmental constraints, firms should implement product and technology innovation rather than passively complying with regulations. Green technology innovation can also improve resource efficiency, reduce production costs, and improve financial performance (\citet{xie2015green}). By implementing green innovations, firms can use alternative energy sources and improve production processes, thereby increasing energy efficiency, reducing pollution, and ensuring compliance with environmental regulations to avoid penalties (\citet{yu2017study}). Liu and Wang (2021) found that although green technology innovation is a high-risk choice, firms can earn a risk premium from it, which earns stock returns. Green technology innovation can also lower the cost for governments and non-governmental organizations to reduce pollution, playing an important role in emission reduction (\citet{carrion2010environmental}). The pollution reduction and improved resource efficiency from green technology innovation contribute to high-quality local development, helping to achieve a win-win situation for environmental protection and business growth (Qi et al., 2018).



\subsection*{1.3 Research on the Relationship between Environmental Regulation and Green Technology Innovation}
The relationship between environmental regulation and green technology innovation has long been an important topic in economics. The main viewpoints are the "compliance cost theory," the "Porter Hypothesis," and the "uncertainty view," with no consensus yet reached.

\paragraph{The "compliance cost theory"} It argues that environmental regulation increases firms' operating costs, crowds out R\&D resources, constrains economic development, and impedes technological innovation. Environmental supervision forces firms to allocate inputs like labor and capital to pollution reduction (\citet{ambec2013porter}). Under intense regulatory pressure, firms may be forced to cut production or even shut down (\citet{petroni2019rethinking}). Local governments' emission reduction targets can reduce corporate cash flow and lower innovation efficiency (\citet{tang2020does}). Tu et al. (2015) found that although the pollution permit trading system reduced inefficiencies in SO2 permit allocation, due to market imperfections, it failed to promote corporate green technology innovation in either the short or long term. An increase in pollution fees will raise firms' compliance costs, squeeze R\&D funds, and negatively affect innovation (Niu and Liu, 2021).

\paragraph{The "Porter Hypothesis,"} Scholars like Michael Porter argued that environmental regulation can stimulate firms to innovate to meet environmental standards. The benefits of this innovation may offset or even exceed the costs, thereby promoting corporate development (\citet{porter1996america}). Environmental regulation is an important tool for encouraging firms to take environmental action (\citet{rugman1998corporate}). In a world of imperfect information, environmental regulation can help firms identify inefficient uses of costly resources. It can also generate and spread new information (e.g., best available technologies), help organizations overcome inertia, and incentivize innovation (\citet{ambec2013porter}). Empirical studies in China have found that the construction of low-carbon cities promotes green technology innovation in high-carbon industries (Xu and Cui, 2020). Similarly, the implementation of the \textit{Ambient Air Quality Standard}, aimed at increasing environmental information transparency, has stimulated green technology innovation in high-environmental-risk industries. Furthermore, stricter environmental law enforcement, media exposure, and effective public supervision can strengthen the positive impact of information disclosure on green innovation (Wang and Wang, 2021).

\paragraph{The "uncertainty view"} It suggests that the impact of environmental regulation on green technology innovation is not straightforward. Because regulation can simultaneously lead to both output-increasing and output-reducing effects, its net impact is difficult to determine (\citet{boyd1999impact}). \citet{lanjouw1996innovation} argued there is no significant correlation between increased emission-reduction expenditures caused by environmental regulation and corporate green technology innovation. \citet{cheng2017emissions} believed that the effectiveness of environmental regulation depends on its intensity. Some empirical studies confirm the existence of a threshold effect. For instance, Cai and Zhou (2017), using total pollution fees to measure the intensity of market-based regulation, found an "inverted U-shaped" relationship with green technology innovation: at low levels, it has a negative impact, but after crossing a certain threshold, it promotes innovation. In contrast, Zhang et al. (2019), using a game theory model and provincial panel data, demonstrated a "U-shaped" relationship between environmental regulation and green technology innovation.

The inconsistent conclusions in existing research may be due to the different effects of heterogeneous regulatory tools (Li and Xiao, 2020). Therefore, although there is extensive literature on environmental regulation, the specific impact of environmental administrative penalties on corporate green technology innovation is still not clear and needs further study.

\newpage
\bibliographystyle{plainnat}
\bibliography{references}


\end{document}