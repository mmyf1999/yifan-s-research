\documentclass[12pt, a4paper]{article}
\usepackage[utf8]{inputenc}
\usepackage[T1]{fontenc}
\usepackage{geometry}
\geometry{left=2.5cm, right=2.5cm, top=2.5cm, bottom=2.5cm}
\usepackage{amsmath, amssymb}
\usepackage{booktabs}
\usepackage{xcolor}
\usepackage{natbib}
\usepackage{graphicx} 
\usepackage{threeparttable}
\usepackage{setspace} 
\usepackage{float}
\usepackage{indentfirst}
\usepackage{threeparttable}
\usepackage{siunitx}
\sisetup{
  detect-mode,
  input-signs = ,
  input-symbols = (),
  table-number-alignment = center,
  round-mode = places,
  round-precision = 3
}
\usepackage{caption}
\usepackage{longtable}
\usepackage{multirow}
\usepackage{parskip} 
\onehalfspacing 
\setlength{\parskip}{0.5em} 
\title{Empirical Analysis of the Impact of Environmental Administrative Penalties on Corporate Green Technology Innovation}
\author{Yifan Mao}
\date{November 2025}

\begin{document}

\maketitle
\begin{center}
\section*{Abstract}
\end{center}

Environmental administrative penalties are an important part of China’s environmental governance system. They can encourage individuals and businesses to comply with environmental protection regulations and incentivize companies to engage in green technological innovation, thereby contributing to our country’s goal of achieving harmonious coexistence between humans and nature. By analyzing the impact and mechanism of environmental administrative penalties on corporate green technological innovation, this dissertation provides theoretical reference and policy suggestions for the government to carry out environmental governance and achieve high-level protection and high-quality development.
This thesis first reviews the policy practice of environmental administrative penalties and finds that since 2018, environmental administrative penalties in China have shown a new characteristic: the average fine has significantly increased. Secondly, based on the cost-benefit theory, the impact and mechanism of environmental administrative penalties on corporate green technological innovation are explored. Thirdly, using the manually compiled environmental administrative penalty data of A-share listed companies from 2015 to 2020, a fixed effects model was used to empirically test the impact of environmental administrative penalties on corporate green technological innovation behaviors. Then we examine the mechanism by which environmental administrative penalties promote corporate green technological innovation. Furthermore, this dissertation compares the differences in the impact of the intensity of environmental administrative penalties and the number of environmental administrative penalties on corporate green technological innovation. and explores the "peer effect" of environmental administrative penalties and the moderating role of government environmental subsidies.

The study found that: First, environmental administrative penalties significantly promote corporate green technological innovation. This innovation effect mainly comes from the increase in the intensity of environmental administrative penalties rather than the increase in the number of environmental administrative penalties. Second, environmental administrative penalties promote corporate green technological innovation by increasing media attention and corporate R\&D investment.Third, environmental administrative penalties also have a "peer effect", which promotes green technological innovation among companies in the same region and industry that have not been punished. Fourth, government environmental subsidies will crowd out the promotion effect of penalties on green technological innovation, and this crowding out effect is more obvious for non-heavy polluting industries.

Based on the research conclusions, this dissertation puts forward the following policy recommendations: improve the environmental supervision mechanism, strictly enforce the law, appropriately increase the intensity of environmental administrative penalties, and prevent companies from "actively breaking the law"; pay attention to the long-term development of companies and establish a "look back" mechanism; improve information disclosure and establish a special reminder mechanism for "peer enterprises" to serve as a warning to others; pay attention to the interaction of policies to ensure that policies work together to promote the long-term sustainable development of enterprises.

\textbf{Key words}: Environmental Administrative Penalties; Green Technological Innovation; Peer Effect

\newpage
\section{Introduction}

\subsection{Research Background}

Over the past 40 years of Reform and Opening-up, China has achieved an economic growth miracle that has attracted worldwide attention. However, the long-standing extensive mode of economic development has aggravated the ecological burden, which is detrimental to the modernization goal of achieving a harmonious coexistence between humanity and nature. General Secretary Xi Jinping emphasized at the Fifth Plenary Session of the 19th Central Committee of the Communist Party of China the need to thoroughly implement the new development philosophy of innovation, coordination, greenness, openness, and sharing. This philosophy highlights the close relationship between economic growth and environmental protection. In the context of the deteriorating global environment, China actively advocates for the construction of an ecological civilization, placing green development in a more prominent position. Particularly at this critical moment of addressing climate change and achieving the goals of carbon peaking and carbon neutrality, green technology innovation is regarded as one of the keys to realizing sustainable economic development.

In recent years, the Chinese government has introduced a series of policy measures to promote green development, including the State Council's \textit{Guidance on Accelerating the Establishment and Improvement of a Green and Low-Carbon Circular Economic System}. These documents emphasize principles such as innovation-driven development, the encouragement of R\&D in green and low-carbon technologies, and the strengthening of the principal position of enterprises in innovation. Green technology innovation is a key factor in achieving the coordinated development of economic growth and environmental protection(\citet{magat1978pollution}). For the ecological environment, when enterprises engage in green technology innovation regarding products and technologies, they can reduce resource consumption, lower environmental pollution, and improve ecological benefits. For the enterprises themselves, green technology innovation allows them not only to reduce production costs and improve product competitiveness but also to actively fulfill their social responsibilities and contribute to the cause of environmental protection (\citet{barney1991firm}).

However, as a large late-developing country, the environmental problems caused by China's traditional production and consumption patterns have not been effectively resolved, and the rule of law in environmental protection urgently needs strengthening. The ecological environment possesses the attributes of a public good; in the absence of supervision, individuals and enterprises lack the motivation to improve it. General Secretary Xi Jinping has emphasized: ``Protecting the ecological environment must rely on institutions and the rule of law.''

China's process of legalizing environmental protection has made significant progress over the past few decades, including the promulgation and implementation of laws and regulations such as the \textit{Environmental Protection Law of the People's Republic of China}, the \textit{Water Pollution Prevention and Control Law}, the \textit{Law on the Prevention and Control of Atmospheric Pollution}, and the \textit{Measures for Environmental Administrative Punishment}. These laws and regulations respectively stipulate the basic principles of environmental protection, the scope of rights and obligations, environmental quality standards, and penalties for environmental violations. 

Nevertheless, while the establishment of an environmental legal system is crucial, its implementation is equally important. As one of the critical means of constructing the rule of law in environmental protection, environmental administrative penalties still face certain problems and challenges during actual execution. On the one hand, some local governments may exhibit issues such as lax environmental enforcement and local protectionism; driven by economic incentives, some grassroots governments may act as ``protective umbrellas'' for polluting enterprises, tacitly approving or even indulging illegal discharge by enterprises within their jurisdiction (\citet{zhang2018does}). On the other hand, some enterprises may demonstrate weak awareness of environmental laws and engage in evasion of law enforcement. When the cost of violations lacks sufficient deterrent force against environmental polluters, enterprises often harbor a mentality of relying on luck and may even choose to actively violate the law. Therefore, how to effectively implement environmental administrative penalties to incentivize enterprises to actively fulfill their environmental responsibilities and carry out green technology innovation has become a pressing practical issue in the field of environmental protection.

It is against this realistic background that this paper aims to deeply investigate the impact of environmental administrative penalties on corporate green technology innovation and explore its mechanisms. The goal is to provide theoretical references and policy recommendations for strengthening the rule of law in environmental protection, promoting corporate green technology innovation, and achieving high-quality economic development.

\subsection{Research Significance}

Based on manually collected data on environmental administrative penalties received by listed companies from 2015 to 2020, this paper focuses on the impact of the intensity of environmental administrative penalties on corporate green technology innovation behavior and its mechanisms. Furthermore, it explores the ``peer effect'' of environmental administrative penalties and their interaction with government environmental subsidies. The significance of this research is mainly reflected in the following two aspects:

\textbf{First, at the theoretical level:} This paper incorporates the external shock of environmental administrative penalties into the ``cost-benefit analysis'' framework to study its impact on corporate green technology innovation. This enriches the analytical framework of ``cost effect analysis,'' provides a new perspective and logical analysis method for research on improving the effectiveness of environmental administrative penalties, and expands the research horizon and ideas regarding environmental administrative penalties. This study also supplements existing literature in the field of environmental regulation and enriches the theory of the mechanism of environmental administrative penalties.

\textbf{Second, at the practical level:} Environmental administrative penalties are an important link in China's environmental protection legal system. However, their execution still faces difficulties and challenges, and the understanding of the role of environmental administrative penalties by some local administrative organs remains superficial. This paper explores the forcing effect (reverse transmission effect) of environmental administrative penalties and its mechanisms from both theoretical and empirical perspectives. Moreover, it compares the differential impacts of penalty intensity versus penalty frequency and explores the interactive effects of different government policies. This study is conducive to deepening the understanding of environmental administrative penalties and provides useful references for improving the scientific governance level of local governments.

\section{Literature Review}

\subsection{Research on Environmental Regulation}
Environmental regulation is the process of controlling and managing the environment through laws, policies, or other administrative means. These regulations aim to protect, maintain, or improve the natural environment to achieve harmony between humans and nature. According to the degree of government control, environmental regulations can be divided into three categories: command-and-control, market-based, and voluntary (Song et al., 2018).

\subsubsection*{1) Command-and-Control Environmental Regulation}
Command-and-control environmental regulation is the dominant environmental management tool in China. Its advantages include clarity, equity, and ease of enforcement(Tan, 2018), and it has played a significant role in controlling pollution and improving environmental quality in the country. Based on the timing of control, this type of regulation can be divided into three categories: pre-control, in-process control, and post-control.

\paragraph{Pre-control} command-and-control regulations include systems like the "Environmental Impact Assessment (EIA)" and the "Three Simultaneities System." The EIA system requires that before a project begins construction, its potential environmental impacts from site selection, design, and future operation must be investigated and evaluated. Preventive measures must be conducted and submitted for legal approval (Wang, 2004). The "Three Simultaneities System" requires that environmental protection facilities must be designed and put into operation at the same time as the main project. These two pre-control measures reflect the policy goal of "preventing environmental pollution."

\paragraph{In-process control} command-and-control regulations include the "emission standards system" and "local government environmental targets." The emission standards system sets detailed technical standards for the concentration and total volume of pollutants that companies can discharge during production. Laws such as China's \textit{Environmental Protection Law}, \textit{Air Pollution Prevention and Control Law}, and \textit{Water Pollution Prevention and Control Law} all have clear requirements for corporate emissions, with standards varying across different industries. This system is the basis for China's environmental quality assessment and management. Raising pollution standards can push companies toward cleaner production and better pollution treatment, thereby reducing pollution levels (Zhang et al., 2023). To reverse the environmental degradation caused by China's extensive growth, the State Council made cadres' performance in environmental governance a criterion for their promotion in 2006. Local governments facing these targets have shown more significant progress in industrial upgrading (Yu et al., 2020), and these pollution controls have helped reduce resource misallocation in polluting industries and improve overall productivity (Han et al., 2020). However, the pressure of environmental targets also conflicts with short-term economic goals. Regions with high emission reduction targets have seen significant declines in their economic growth and fiscal revenue(Xie and Wang, 2022).

\paragraph{Post-control} command-and-control regulations include environmental administrative penalties and environment remediation liability systems. The rules from pre-control and in-process control provide the basis for imposing administrative penalties, while these penalties, in turn, ensure that companies and individuals comply with environmental regulations (Tan, 2018). Research on environment remediation liability mainly concentrates in the law field. This system provides an institutional guarantee for restoring ecosystems after damage, promoting harmony between humans and nature, and building a harmonious society (Li, 2013).

\subsubsection*{2) Market-Based Environmental Regulation}
Unlike command-and-control regulations, market-based regulations incentivize rather than force firms. They can be divided into Coasean and Pigouvian approaches, depending on whether they create a market or use the existing market.

\paragraph{Coasean approaches} Ronald Coase suggested that if property rights are well-defined and transaction costs are low enough, market transactions can lead to an optimal allocation of resources. Coasean tools mainly include carbon emissions trading, pollution discharge permit trading, and energy use rights trading. A carbon emissions trading policy first sets a maximum allowable level of carbon emissions for a region and then allocates emission permits, thereby controlling the total regional emissions. Companies that reduce their emissions can gain revenue by selling their surplus permits. This system can adjust the energy structure, improve regional technology levels, and thus achieve carbon reduction and green development (Dong and Wang, 2021). Pollution permit trading systems improve marketization and the development of factor markets by clearly defining property rights, which helps reduce regional energy consumption, promote green technology innovation, and improve green total-factor energy efficiency (Shi and Li, 2020). Energy use rights trading can control total energy consumption and offers significant economic and energy-saving potential for industries. However, this potential varies across industries, and in some sectors, energy-saving potential might be crowded out (Zhang and Zhang, 2019).

\paragraph{Pigouvian approaches} Arthur Pigou argued that market failures occur due to externalities. The government can correct these failures by imposing taxes or providing subsidies to align private costs with social costs, thereby maintaining market efficiency. Pigouvian tools mainly include the environmental protection tax and government subsidies. China's \textit{Environmental Protection Tax Law}, implemented on January 1, 2018, internalizes the external costs of pollution by setting tax standards for different pollutants, encouraging companies to reduce emissions and innovate. The environmental tax can improve the efficiency of fossil fuel use and incentivize green technology innovation aimed at pollution reduction (Liu and Xiao, 2022). A carbon tax might reduce output in the short term, but the impact is not substantial, and its long-term emission reduction effect is significant (Chen, 2011). There is an ongoing debate about whether government subsidies "stimulate" or "weaken" corporate innovation (Guo, 2018). The "stimulation" guys argue that subsidies can correct market failures in the innovation process and encourage companies to invest in R\&D (\citet{jaffe2015impact}). The "weakening" guys argue that information asymmetry between the government and firms can lead to over-investment (Wei et al., 2015). Furthermore, since environmental subsidies often target environmental construction projects rather than green technology innovation itself, opportunistic behavior by firms might "weaken" their motivation for green innovation (Li and Xiao, 2020).

\subsubsection*{3) Voluntary and Persuasive Regulation}
In a narrow sense, voluntary regulation refers to influencing environmental protection behavior through moral persuasion. In a broader sense, it includes all environmental regulation tools other than command-and-control and market-based types. Examples include environmental information disclosure and ESG (Environmental, Social, and Governance) ratings.

Environmental information disclosure can increase information symmetry, promote local green technology innovation and industrial structure upgrading, and thus achieve pollution and carbon reduction (Shao and Wang, 2024). Public disclosure of environmental information also affects a company's market value. After green rankings are published, the market value of top-ranked companies tends to rise, while that of bottom-ranked companies falls. The enforcement of laws like the \textit{Clean Air Act} has made the impact of environmental legal information on corporate market value even more obvious (Aaron et al., 2012; Badrinath and Bolster, 1996). ESG serves as a supplement to formal environmental regulation and is an important international standard for a company's green and sustainable development. Companies with good ESG performance typically have higher quality information disclosure and significantly lower financing costs (Qiu and Yin, 2019). Excellent ESG performance can also reduce a company's information and operational risks, leading to lower audit fees (Xiao et al., 2021). By easing financing constraints and improving employees' innovation and risk-taking, good ESG performance ultimately promotes corporate innovation (Fang and Hu, 2023).

To summarize, command-and-control regulation is highly deterministic and uses governmental authority to address environmental externalities. Given that the environment is a "public good," this type of regulation is essential. However, it can also lead to "government failure" due to insufficient information and a lack of flexibility, as it often applies a one-size-fits-all standard. Market-based regulation is more economically efficient and better at promoting continuous environmental improvement, but it faces challenges in pricing, market creation, and potential market failures. Voluntary regulation holds great promise and is highly flexible, but it relies on moral values and choices, which take a long time to foster.

\subsection{Research on Green Technology Innovation}

In the early 1990s, academia began to focus on the negative environmental impacts of economic development and recognized the positive role of corporate innovation in improving the environment. However, there is still no consensus on the definition of green innovation. \citet{peter1997sustainability} defined green innovation as "new or modified processes, techniques, systems, and products to avoid or reduce environmental damage." Dong (2010), following the concept of the green economy, described green innovation as a complete process that includes innovation at the end-of-pipe, innovation in the production process, innovation in products, and innovation at the system level. \citet{driessen2013green} expanded the concept further, arguing that green innovation should not merely aim to reduce environmental burdens but should aim to produce significant environmental benefits. Current research often identifies corporate green technology innovation using the International Patent Classification (IPC) Green Inventory, published by the World Intellectual Property Organization (WIPO) in 2010 (Qi et al., 2018; Li and Xiao, 2020; Liu and Xiao, 2022). The WIPO Green Inventory covers seven main categories: alternative energy production, transportation, energy conservation, waste management, agriculture/forestry, administrative/regulatory/design aspects, and nuclear power generation. This provides a comprehensive way to identify green technology innovation. This paper adopts the WIPO classification to identify green technology innovation.

Green technology innovation can have a long-term impact on individual firms and the macroeconomy. Firms with more green technology innovations can produce more green products, differentiate themselves from competitors, and gain a green competitive advantage (\citet{barney1991firm}). \citet{clarke1994challenge} argued that when facing environmental constraints, firms should implement product and technology innovation rather than passively complying with regulations. Green technology innovation can also improve resource efficiency, reduce production costs, and improve financial performance (\citet{xie2015green}). By implementing green innovations, firms can use alternative energy sources and improve production processes, thereby increasing energy efficiency, reducing pollution, and ensuring compliance with environmental regulations to avoid penalties (\citet{yu2017study}). Liu and Wang (2021) found that although green technology innovation is a high-risk choice, firms can earn a risk premium from it, which earns stock returns. Green technology innovation can also lower the cost for governments and non-governmental organizations to reduce pollution, playing an important role in emission reduction (\citet{carrion2010environmental}). The pollution reduction and improved resource efficiency from green technology innovation contribute to high-quality local development, helping to achieve a win-win situation for environmental protection and business growth (Qi et al., 2018).



\subsection{Research on the Relationship between Environmental Regulation and Green Technology Innovation}
The relationship between environmental regulation and green technology innovation has long been an important topic in economics. The main viewpoints are the "compliance cost theory," the "Porter Hypothesis," and the "uncertainty view," with no consensus yet reached.

\paragraph{The "compliance cost theory"} It argues that environmental regulation increases firms' operating costs, crowds out R\&D resources, constrains economic development, and impedes technological innovation. Environmental supervision forces firms to allocate inputs like labor and capital to pollution reduction (\citet{ambec2013porter}). Under intense regulatory pressure, firms may be forced to cut production or even shut down (\citet{petroni2019rethinking}). Local governments' emission reduction targets can reduce corporate cash flow and lower innovation efficiency (\citet{tang2020does}). Tu et al. (2015) found that although the pollution permit trading system reduced inefficiencies in SO2 permit allocation, due to market imperfections, it failed to promote corporate green technology innovation in either the short or long term. An increase in pollution fees will raise firms' compliance costs, squeeze R\&D funds, and negatively affect innovation (Niu and Liu, 2021).

\paragraph{The "Porter Hypothesis,"} Scholars like Michael Porter argued that environmental regulation can stimulate firms to innovate to meet environmental standards. The benefits of this innovation may offset or even exceed the costs, thereby promoting corporate development (\citet{porter1996america}). Environmental regulation is an important tool for encouraging firms to take environmental action (\citet{rugman1998corporate}). In a world of imperfect information, environmental regulation can help firms identify inefficient uses of costly resources. It can also generate and spread new information (e.g., best available technologies), help organizations overcome inertia, and incentivize innovation (\citet{ambec2013porter}). Empirical studies in China have found that the construction of low-carbon cities promotes green technology innovation in high-carbon industries (Xu and Cui, 2020). Similarly, the implementation of the \textit{Ambient Air Quality Standard}, aimed at increasing environmental information transparency, has stimulated green technology innovation in high-environmental-risk industries. Furthermore, stricter environmental law enforcement, media exposure, and effective public supervision can strengthen the positive impact of information disclosure on green innovation (Wang and Wang, 2021).

\paragraph{The "uncertainty view"} It suggests that the impact of environmental regulation on green technology innovation is not straightforward. Because regulation can simultaneously lead to both output-increasing and output-reducing effects, its net impact is difficult to determine (\citet{boyd1999impact}). \citet{lanjouw1996innovation} argued there is no significant correlation between increased emission-reduction expenditures caused by environmental regulation and corporate green technology innovation. \citet{cheng2017emissions} believed that the effectiveness of environmental regulation depends on its intensity. Some empirical studies confirm the existence of a threshold effect. For instance, Cai and Zhou (2017), using total pollution fees to measure the intensity of market-based regulation, found an "inverted U-shaped" relationship with green technology innovation: at low levels, it has a negative impact, but after crossing a certain threshold, it promotes innovation. In contrast, Zhang et al. (2019), using a game theory model and provincial panel data, demonstrated a "U-shaped" relationship between environmental regulation and green technology innovation.

The inconsistent conclusions in existing research may be due to the different effects of heterogeneous regulatory tools (Li and Xiao, 2020). Therefore, although there is extensive literature on environmental regulation, the specific impact of environmental administrative penalties on corporate green technology innovation is still not clear and needs further study.



\section{Hypotheses}

Based on the theoretical analysis of the characteristics of environmental administrative penalties, the Cost-Benefit Theory, and the Porter Hypothesis, we propose the following research hypotheses.

\subsection{The Forcing Effect of Penalties}
According to the Cost-Benefit Theory, companies make decisions to maximize profits. When facing environmental administrative penalties, companies weigh the "direct costs" (fines) and "indirect costs" (reputation loss) against the "compliance costs" and "transformation costs" of innovation.

The Porter Hypothesis suggests that appropriate environmental regulations can force enterprises to carry out green technology innovation. Innovation can bring compensation benefits, such as higher production efficiency and better product competitiveness. When the intensity of penalties is high enough, the cost of violation exceeds the cost of compliance. To achieve long-term profit maximization, enterprises will choose to carry out green technology innovation.

\begin{quote}
    \textbf{Hypothesis 1 (H1):} Environmental administrative penalties drive corporate green technology innovation.
\end{quote}

\subsection{Transmission Mechanisms: Media and R\&D}
Environmental administrative penalties exert external pressure and drive internal resource reallocation.

First, regarding external pressure, penalties act as negative news that attracts media attention. Media coverage increases the "indirect costs" of violation (e.g., damage to corporate image). To restore their reputation and respond to stakeholder concerns, companies are motivated to improve their environmental performance through innovation.

\begin{quote}
    \textbf{Hypothesis 2 (H2):} Environmental administrative penalties drive corporate green technology innovation by increasing media attention.
\end{quote}

Second, regarding internal resources, penalties signal that existing technologies are insufficient to meet environmental standards. To solve the pollution problem fundamentally, companies need to allocate more resources to research and development (R\&D). Increasing R\&D investment and personnel allows companies to update their knowledge systems and achieve green transformation.

\begin{quote}
    \textbf{Hypothesis 3 (H3):} Environmental administrative penalties drive corporate green technology innovation by increasing corporate R\&D investment.
\end{quote}

\subsection{Peer Effects}
According to social psychology and signaling theory, the decision-making of an enterprise is influenced by its peers. When the government imposes penalties on a violating enterprise, it sends a signal of strict law enforcement to other companies in the same region or industry.

Observing the high costs paid by the penalized firm, peer enterprises will adjust their expectations of illegal costs and benefits. To avoid facing similar penalties in the future, peer enterprises will take proactive measures and carry out green technology innovation.

\begin{quote}
    \textbf{Hypothesis 4 (H4):} The intensity of environmental administrative penalties has a "peer effect," promoting green technology innovation in peer enterprises.
\end{quote}


\section{Empirical Design}

\subsection{Baseline Model}
Referencing the research of Guo et al. (2023) and Li Qingyuan and Xiao Zehua (2020), the baseline regression model is set as follows:

\begin{equation} 
    Tgreen_{i,t} = \beta_0 + \beta_1 fine_{i,t} + \beta_n \mathbb{X}_{i,t} + \sum Area_{i,t} + \sum Year_{i,t} + \sum Ind_{i,t} + \varepsilon_{i,t}
\end{equation}

$Tgreen_{i,t}$ represents the green technology innovation ($Tgreen$) of listed company $i$ in year $t$. Considering that the impact of environmental administrative penalties on green technology innovation may not happen immediately, and there is a certain time lag from R\&D investment to patent output, we added the regression results of the green technology innovation level one period ahead ($Tgreen_{i,t+1}$) to ensure the robustness of the research. 

$fine_{i,t}$ represents the intensity of environmental administrative penalties received by listed company $i$ in year $t$. $\mathbb{X}_{i,t}$ represents the set of control variables. Considering that environmental management measures in different regions and times may change, $Area$ and $Year$ are provincial and annual dummy variables to control for time and regional fixed effects. Considering that the capital, labor, and production factors of different industries vary greatly, and their responses to environmental administrative penalties differ, $Ind$ is an industry dummy variable to control for industry fixed effects. The industry classification is based on the "Guidelines for the Industry Classification of Listed Companies (2012 Revision)". To ensure the robustness of the research results, this paper also uses the Heckman two-step method and Propensity Score Matching (PSM) method for robustness tests.

\subsection{Sample Selection and Data Sources}
This paper takes the environmental administrative penalty data of China's A-share listed companies from 2015 to 2020 as the sample. Limited by data availability, we selected environmental administrative penalty data from 2015 onwards. Information on environmental administrative penalties for listed companies is concentrated in annual reports or corporate social responsibility reports, but some listed companies may choose not to disclose it. The objectivity and timeliness of the information are difficult to guarantee (Liu Liya et al., 2022), which is unfavorable for identifying the impact of environmental legal risks on listed companies. With the establishment and improvement of the environmental information disclosure system, environmental administrative penalty information has been gradually included in the scope of government environmental information disclosure. The environmental administrative penalty information disclosed by the government has better objectivity and higher quality.

The raw data on environmental administrative penalties for listed companies in this paper were compiled by "Qingyue Data", a council member unit of the Green Finance Committee of the China Society for Finance and Banking, based on environmental administrative penalty decision letters disclosed by government departments at all levels. Through manual sorting by the author, the environmental administrative penalty data were matched with the financial and patent data of listed companies to obtain the data on environmental administrative penalties of China's A-share listed companies from 2015 to 2020. The financial data of listed companies comes from the Wind database and CSMAR database. This paper processed the raw data as follows: (1) Exclude enterprises that were treated as ST (Special Treatment) during the sample period; (2) Delete samples with missing variable observations; (3) Exclude financial industry samples. Summing up the environmental administrative penalty amounts of enterprises in the same year, we obtained 5,023 observation samples.

\subsection{Variable Descriptions}

\subsubsection{Environmental Administrative Penalty Intensity}
Given that the scale of listed companies varies greatly, larger companies have stronger risk resistance capabilities, and the impact of the same amount of environmental administrative penalty is smaller. Unlike the conventional measurement method of penalty frequency, this paper uses the ratio of the total amount of environmental administrative penalties received by the listed company in the current year to the total assets of the listed company to represent the intensity of environmental administrative penalties. To increase readability, the ratio is multiplied by one thousand to obtain the environmental administrative penalty intensity indicator ($fine$). The larger the indicator, the greater the intensity of environmental administrative penalties. At the same time, other standardization methods are used for robustness tests. In the further analysis, a comparative analysis is conducted on whether enterprises are more concerned about the number of penalties or the intensity of penalties.

\subsubsection{Dependent Variable: Green Technology Innovation}
Referencing the research of Li Qingyuan and Xiao Zehua (2020), we first retrieved the patent application and authorization status of listed companies from the State Intellectual Property Office of China (SIPO). Then, we used the IPC classification codes for green patents launched by WIPO in 2010 for matching to obtain the number of green technology innovation patents applied for by listed companies and their subsidiaries in the current year. After adding 1, the logarithm is taken to obtain the indicator measuring corporate green technology innovation ($Tgreen$). The larger the indicator, the higher the corporate green technology innovation capability.

\subsubsection{Control Variables}
Drawing on the research of Li Qingyuan and Xiao Zehua (2020), Zhang Qi and Zou Mengqi (2022), and Qi Shaozhou et al. (2018), this paper selects the following 10 control variables that may affect the green technology innovation of listed companies. Variable definitions are shown in Table 1.

\begin{table}[H]
    \centering
    \caption{Control Variable Definitions}
    \footnotesize
    \begin{tabular}{lll}
        \toprule
        \textbf{Variable Definition} & \textbf{Name} & \textbf{Measurement Method} \\
        \midrule
        Firm Size & $lnSize$ & Natural logarithm of total assets \\
        Capital Structure & $Lev$ & Total liabilities / Total assets \\
        Cash Flow Ratio & $Cfo$ & Net cash flow from operating activities / Total assets \\
        Firm Growth & $Growth$ & (Current revenue - Prior revenue) / Prior revenue \\
        Historical Performance & $Lroa$ & Prior net profit / Prior total assets \\
        Market Power & $lnMarket$ & ln(Sales revenue / Operating costs) \\
        Capital Intensity & $lnDensity$ & ln(Total fixed assets / Number of employees) \\
        Management Incentive & $Share$ & Management shareholding / Total share capital $\times$ 100\% \\
        CEO Public Background & $PC$ & 1 if CEO served in government, 0 otherwise \\
        Listing Age & $lnAge$ & ln(Years of listing + 1) \\
        \bottomrule
    \end{tabular}
\end{table}

\subsubsection{Descriptive Statistics}
The sample of this paper is the panel data of 914 enterprises that received environmental administrative penalties during the period 2015-2020. Since some enterprises went public during the period, it is an unbalanced panel. There are a total of 5,023 observations, of which the total annual fine amount for 1,456 observations is greater than 0. The descriptive statistics of the main variables are shown in Table 2. The mean of $fine$ is 0.013, the minimum value is 0, and the maximum value is 3.885. This is because some enterprises did not receive environmental administrative penalties in some years, while a small number of enterprises received fines amounting to hundreds of millions of yuan in the current year. The statistical values of the remaining variables are basically consistent with existing research and conform to the actual situation in China.

\begin{table}[H]
    \centering
    \caption{Descriptive Statistics}
    \footnotesize
    \begin{tabular}{lcccccc}
        \toprule
        Type & Name & Var & Mean & Min & Max & SD \\
        \midrule
        Dependent & Green Tech Innovation & $Tgreen$ & 1.491 & 0 & 7.782 & 1.432 \\
        Independent & Penalty Intensity & $fine$ & 0.013 & 0 & 3.885 & 0.088 \\
        Controls & Firm Size & $lnSize$ & 23.022 & 19.53 & 29.95 & 1.491 \\
         & Capital Structure & $Lev$ & 0.484 & 0.014 & 1.181 & 0.191 \\
         & Cash Flow Ratio & $Cfo$ & 0.054 & -0.362 & 0.652 & 0.064 \\
         & Firm Growth & $Growth$ & 0.17 & -0.481 & 2.199 & 0.356 \\
         & Historical Performance & $Lroa$ & 0.04 & -0.593 & 0.655 & 0.056 \\
         & Market Power & $lnMarket$ & 4.291 & -1.775 & 4.525 & 0.278 \\
         & Capital Intensity & $lnDensity$ & 14.621 & 10.87 & 18.7 & 0.9 \\
         & Mgmt Incentive & $Share$ & 10.231 & 0 & 100 & 17.376 \\
         & CEO Background & $PC$ & 0.293 & 0 & 1 & 0.455 \\
         & Listing Age & $lnAge$ & 2.396 & 0 & 3.434 & 0.746 \\
        \bottomrule
    \end{tabular}
\end{table}

\section{Baseline Regression Results}

The baseline regression results are shown in Table 3. The impact of environmental administrative penalty intensity ($fine$) on green technology innovation is significantly positive at the 1\% level. Leading the dependent variable by one year did not change the regression results. This indicates that an increase in the intensity of environmental administrative penalties can promote corporate green technology innovation. The Variance Inflation Factor (VIF) test for the variables in the main regression (3) found that the VIF values of all variables were less than 2.5, far less than 10, so there is no serious multicollinearity problem in the regression. Hypothesis H1 is verified.

Taking column (3) as an example, if the intensity of environmental administrative penalties increases by one standard deviation, corporate green technology innovation increases by 2.92\% ($0.422 \times 0.103 / 1.491$). The impact of environmental administrative penalties on corporate green technology innovation is consistent with the expectation of the "Porter Hypothesis". When the intensity of environmental administrative penalties faced by an enterprise is large, the pressure of "direct costs" faced by the enterprise and the external pressure brought by laws and public opinion make the enterprise realize the unsustainability of illegal behavior. The enterprise carries out green technology innovation to maintain a good social image. At the same time, corporate managers also realize the government's determination to protect the environment and carry out green technology innovation to achieve future "profit maximization". With the green technology innovation variable led by one period, the impact of environmental administrative penalty intensity remains significant. On the one hand, this illustrates the robustness of the results; on the other hand, it also illustrates that the cost pressure brought by the increase in environmental administrative penalty intensity and the expectation of future "profit maximization" are persistent, urging enterprises to continue to carry out green technology innovation activities.

\begin{table}[H]
    \centering
    \caption{Impact of Environmental Administrative Penalties on Corporate Green Technology Innovation}
    \footnotesize
    \begin{tabular}{lcccccc}
        \toprule
        Variable & (1) & (2) & (3) & (4) & (5) & (6) \\
         & $Tgreen$ & $Tgreen$ & $Tgreen$ & $F.Tgreen$ & $F.Tgreen$ & $F.Tgreen$ \\
        \midrule
        $fine$ & 0.564*** & 0.446*** & 0.422*** & 0.489*** & 0.435*** & 0.420*** \\
         & (0.102) & (0.114) & (0.103) & (0.087) & (0.094) & (0.093) \\
        Constant & -8.266*** & -6.890*** & -11.398*** & -8.262*** & -7.293*** & -11.714*** \\
         & (0.870) & (0.914) & (0.855) & (0.837) & (0.895) & (0.841) \\
        \midrule
        Controls & Yes & Yes & Yes & Yes & Yes & Yes \\
        Region FE & No & Yes & Yes & No & Yes & Yes \\
        Time FE & No & Yes & Yes & No & Yes & Yes \\
        Industry FE & No & No & Yes & No & No & Yes \\
        Observations & 5023 & 5023 & 5023 & 4785 & 4785 & 4785 \\
        $R^2$ & 0.275 & 0.317 & 0.536 & 0.277 & 0.314 & 0.521 \\
        \bottomrule
        \multicolumn{7}{l}{\scriptsize Note: *, **, *** denote significance at 10\%, 5\%, 1\% levels. Parentheses contain standard errors clustered at firm level.}
    \end{tabular}
\end{table}

\section{Robustness Checks}

\subsection{Endogeneity Treatment}
\textbf{Heckman Two-Step Method.} The raw data on environmental administrative penalties in this paper comes from government portal websites. However, due to local protectionism and potential issues with website update efficiency, there is a possibility of incomplete disclosure of environmental administrative penalty information on government portals. This leads to non-random sample collection, which may cause endogeneity due to sample selection bias. Here we use the Heckman two-step method to exclude the endogeneity problem that may be caused by sample selection bias.

In the first stage of the Heckman two-step method, we use the Probit model regression. The dependent variable is whether the enterprise received an environmental administrative penalty in the current year: $fine\_dummy$. Since a certain decision of an enterprise is easily influenced by other listed companies in the same region and industry observed recently (Kaustia and Rantala, 2015), according to the "peer influence theory", when misconduct is pointed out as unethical and faces penalties from regulatory authorities, peer managers will tend to reduce misconduct (Gino et al., 2009). When an enterprise faces environmental administrative penalties, its peers receive a "deterrent signal", and peer enterprises' perception of environmental violation risks and costs continues to rise (Fan Ziying and Zhao Renjie, 2019), thereby reducing the possibility of violation, but the transmission of the signal takes time. Therefore, the mean of environmental administrative penalty intensity received by other enterprises in the same industry and region in the previous year ($L.fine\_partner$) is calculated as the instrumental variable for $fine\_dummy$.

In the second stage, we add the Inverse Mills Ratio obtained in the first stage to equation(1) for regression. The results in Table 4 show that the regression coefficient of the instrumental variable in the first stage is significantly negative at the 1\% level, meaning that when enterprises observe peer penalties, they reduce environmental violations. In the second stage, the regression coefficient of $fine$ is significantly positive at the 1\% level. The Inverse Mills Ratio is not significant, indicating that the sample selection bias is not obvious and has little impact on the main regression. To further exclude the impact that sample selection bias might bring, this paper also removes samples where the observed value of the environmental administrative penalty amount in the current year is 0, keeping only samples where the environmental administrative penalty is greater than 0. Using equation(1) for regression testing again, the coefficient of $fine$ is still significantly positive at the 1\% level. Therefore, after controlling for endogeneity issues, the conclusion that environmental administrative penalties promote corporate green technology innovation is still supported.

\begin{table}[H]
    \centering
    \caption{Endogeneity Treatment: Heckman Two-Step Method}
    \footnotesize
    \begin{tabular}{lc}
        \toprule
         & (1) Heckman \\
        \midrule
        \textbf{First Stage Regression} & $fine\_dummy$ \\
        $L.fine\_partner$ & -2.039*** \\
         & (0.584) \\
        Constant & -2.89*** \\
         & (0.859) \\
        Controls/FE & Yes \\
        \midrule
        \textbf{Second Stage Regression} & $Tgreen$ \\
        $fine$ & 0.490*** \\
         & (0.188) \\
        lambda & 0.0835 \\
         & (0.331) \\
        Constant & -12.923*** \\
         & (1.380) \\
        Controls/FE & Yes \\
        \midrule
        Observations & 3873 \\
        Wald-chi2 & 1836.92 \\
        \bottomrule
    \end{tabular}
\end{table}

\subsection{Propensity Score Matching (PSM)}
Referencing the research of Wang Huiling and Kong Rong (2019) and Liu et al. (2021), the Propensity Score Matching method is adopted to enhance the comparability between the treatment group and the control group and verify the robustness of the research results. We construct a dummy variable $fine\_dummy$ for whether the enterprise received an environmental administrative penalty. If the amount of environmental administrative penalty received by the enterprise is greater than 0, $fine\_dummy=1$, otherwise it equals 0; subsequently, a logit model is used to regress $fine\_dummy$ with all control variables and whether the enterprise is in a heavy pollution industry ($pollution$) as covariates to calculate the propensity score.

There is no consensus in the academic community on which matching method is the best. If multiple matching methods yield consistent conclusions, the robustness of the results can be proven (Wang Huiling and Kong Rong, 2019). Therefore, this paper selects 5 mainstream matching methods. The comparison of sample characteristics before and after k (k=5) nearest neighbor matching is shown in Table 5. The results of other matches are limited by space, and Table 6 shows the overall matching balance test results.

\begin{table}[H]
    \centering
    \caption{Comparison of Sample Characteristics Before and After k (k=5) Nearest Neighbor Matching}
    \footnotesize
    \begin{tabular}{lccccc}
        \toprule
        Variable & Sample & Mean (Treated) & Mean (Control) & Bias & t-value \\
        \midrule
        $lnSize$ & Unmatched & 23.393 & 22.822 & 38.8 & 12.95*** \\
         & Matched & 23.393 & 23.404 & -0.8 & -0.19 \\
        $Lev$ & Matched & 0.50532 & 0.50199 & 1.8 & 0.47 \\
        $Cfo$ & Matched & 0.05979 & 0.06093 & -1.8 & -0.48 \\
        $pollution$ & Matched & 0.40865 & 0.40096 & 1.6 & 0.42 \\
        \bottomrule
    \end{tabular}
\end{table}

\textbf{Balance Test:} A good matching method should solve the balance problem between the treatment group and the control group. The balance test results are shown in Table 6. After matching, the standardized bias of the explanatory variables decreased significantly. The total bias was significantly reduced and was less than the 20\% red line standard stipulated by the balance test. The pseudo R2 and LR statistics decreased significantly. It can be seen that using the PSM propensity score matching method effectively reduced the distribution difference of explanatory variables between the control group and the treatment group, eliminating the estimation bias caused by sample self-selection bias.

\begin{table}[H]
    \centering
    \caption{Balance Test Results of Explanatory Variables Before and After PSM}
    \footnotesize
    \begin{tabular}{lccc}
        \toprule
        Matching Method & Pseudo R2 & LR Stat & Std Bias (\%) \\
        \midrule
        Unmatched & 0.031 & 188.93 & 12.9 \\
        k-Nearest (k=5) & 0.001 & 3.21 & 1.3 \\
        Caliper & 0.001 & 3.77 & 1.4 \\
        Caliper within k & 0.000 & 1.72 & 1.0 \\
        Kernel & 0.000 & 1.72 & 1.0 \\
        Spline & 0.003 & 10.51 & 2.7 \\
        \bottomrule
    \end{tabular}
\end{table}

\textbf{Impact Effect Calculation:} The Average Treatment Effect on the Treated (ATT) of the treatment group after matching is shown in Table 7. The econometric results obtained using the 5 matching methods are basically consistent. The treatment group and the control group have a significant positive difference at the 1\% significance level, which further proves the robustness of the baseline results.

\begin{table}[H]
    \centering
    \caption{Treatment Effect of Propensity Score Matching}
    \footnotesize
    \begin{tabular}{lccc}
        \toprule
        Matching Method & ATT & SE & T-stat \\
        \midrule
        k-Nearest (k=5) & 0.174*** & 0.052 & 3.35 \\
        Caliper (0.020) & 0.173*** & 0.052 & 3.32 \\
        Caliper within k & 0.158*** & 0.048 & 3.25 \\
        Kernel & 0.180*** & 0.048 & 3.73 \\
        Spline & 0.152*** & 0.043 & 3.53 \\
        Mean & 0.167 & - & - \\
        \bottomrule
    \end{tabular}
\end{table}

\subsection{Excluding Competitive Policy Interference}
This part excludes other policies that may affect the green technology innovation of listed enterprises to ensure the robustness of the regression results.

First, the Environmental Protection Tax. Referencing the research of Liu Jinke and Xiao Yiyang (2022), to identify the potential impact of the Environmental Protection Tax, a triple difference interaction term ($Tax_{r,j,t}$) of environmental protection tax regional dummy variable ($Reform_r$) $\times$ industry pollution characteristic variable ($Polluted_j$) $\times$ time dummy variable ($Post_t$) is added to the regression equation. In column (1), $Tax_{r,j,t}$ is used as a control variable; in column (2), samples with $Tax_{r,j,t}=1$ are excluded to eliminate the impact of the implementation of the Environmental Protection Tax on corporate green technology innovation. Specifically, after the implementation of the "Environmental Protection Tax Law of the People's Republic of China", $Reform_r=1$ for regions where local pollution taxes and fees increased, otherwise $Reform_r=0$. If the industry to which the enterprise belongs is a heavy pollution industry, then $Polluted_j=1$, otherwise it is 0. The implementation time of the environmental protection tax is 2018, so $Post_t=1$ for 2018 and after, and 0 before.

Second, the pollution rights trading system implemented in 2007 and the carbon emissions trading system piloted in 2011. Since the years are both before 2015, they cannot be used as control variables. Columns (3) and (4) exclude samples from pollution rights trading pilot regions and carbon emissions trading pilot regions, respectively, to exclude the impact of other policies on environmental administrative penalties.

The results of excluding these three competitive policies are shown in Table 8. The coefficient of $fine$ is always positively significant at least at the 5\% level. This indicates that competitive policies did not affect the significance of the baseline regression, verifying the robustness of the research results.

\begin{table}[H]
    \centering
    \caption{Regression Results Excluding Competitive Policy Effects}
    \footnotesize
    \begin{tabular}{lcccc}
        \toprule
        Variable & (1) & (2) & (3) & (4) \\
         & $Tgreen$ & $Tgreen$ & $Tgreen$ & $Tgreen$ \\
        \midrule
        $fine$ & 0.423*** & 0.437*** & 0.278** & 0.308** \\
         & (0.104) & (0.138) & (0.112) & (0.122) \\
        $Tax$ & -0.049 & & & \\
         & (0.052) & & & \\
        Constant & -11.389*** & -10.714*** & -11.235*** & -11.403*** \\
         & (0.855) & (0.899) & (1.003) & (1.068) \\
        Controls/FE & Yes & Yes & Yes & Yes \\
        Observations & 5023 & 3940 & 2983 & 3053 \\
        $R^2$ & 0.536 & 0.500 & 0.591 & 0.444 \\
        \bottomrule
    \end{tabular}
\end{table}

\subsection{Replacing Variable Measurement Methods}
First, we change the measurement method of the dependent variable ($Tgreen$).
(1) Referencing the research of Wang Banban and Qi Shaozhou (2016), Liu Jinke and Xiao Yiyang (2022), we use the ratio of the total number of green patents applied for by the enterprise in the current year to the total number of invention patents and utility model patents applied for ($Greenrate1$), and the ratio of the total number of green patents obtained by the enterprise in the current year to the total number of invention patents and utility model patents obtained ($Greenrate2$) as dependent variables and use equation(1) for regression.
(2) Referencing the research of Li Qingyuan and Xiao Zehua (2020), we use the logarithm of (1 + number of green patents authorized by the enterprise in the current year) ($Tgreen2$) as the dependent variable and use equation(1) for regression. Remeasuring the impact of environmental administrative penalty intensity on corporate green technology innovation, the results show that no matter how the measurement method of the dependent variable is changed, the conclusion remains robust.

Then, we change the measurement method of the independent variable ($fine$).
(1) Referencing the research of Yu Lianchao et al. (2019) on environmental protection tax, we use the ratio of environmental administrative penalty amount to operating revenue to measure environmental administrative penalty intensity ($fine1$).
(2) We use the number of environmental administrative penalties received by the enterprise in the current year to measure environmental administrative penalty intensity ($fine2$). The number of penalties for an enterprise is related to whether the enterprise complies with environmental protection regulations. For the same enterprise, the more penalties, the more total costs it usually pays. However, the number of penalties is unrelated to the other financial performance of the enterprise, thereby excluding the impact of corporate financial factors on penalty performance.
(3) If the total amount of environmental administrative penalties received by the enterprise in the current year is greater than the total amount of environmental administrative penalties in the previous year, then $fine3=1$, otherwise $fine3=0$. This measures the environmental administrative penalty intensity. The growth of the amount of environmental administrative penalties received by the enterprise makes the enterprise's expectation of high illegal costs in the future clearer.

Table 9 results show that no matter how the measurement method of the independent variable is changed, the regression coefficient of environmental administrative penalty intensity is significantly positive at the 5\% level, which is consistent with the conclusion of this paper.

\begin{table}[H]
    \centering
    \caption{Replacing Variable Measurement Methods}
    \footnotesize
    \begin{tabular}{lcccccc}
        \toprule
        Variable & (1) & (2) & (3) & (4) & (5) & (6) \\
         & $Gr\_rate1$ & $Gr\_rate2$ & $Tgreen2$ & $Tgreen$ & $Tgreen$ & $Tgreen$ \\
        \midrule
        $fine$ & 0.030* & 0.050* & 0.395*** & & & \\
         & (0.018) & (0.028) & (0.115) & & & \\
        $fine1$ & & & & 0.239*** & & \\
         & & & & (0.057) & & \\
        $fine2$ & & & & & 0.005** & \\
         & & & & & (0.003) & \\
        $fine3$ & & & & & & 0.137** \\
         & & & & & & (0.057) \\
        Constant & -0.259*** & -0.228** & -9.166*** & -11.391*** & -11.287*** & -11.251*** \\
        Controls/FE & Yes & Yes & Yes & Yes & Yes & Yes \\
        Obs & 5023 & 5023 & 5023 & 5023 & 5023 & 5023 \\
        $R^2$ & 0.282 & 0.275 & 0.527 & 0.536 & 0.536 & 0.536 \\
        \bottomrule
    \end{tabular}
\end{table}

\subsection{Excluding the Influence of Corporate Strategy Selection}
For enterprises that are implementing strategic expansion and opening up business maps, faced with new business development needs, even if they do not face environmental administrative penalties, they will increase the level of green technology innovation. The background of the corporate CEO also has an impact on the choice of corporate green technology innovation strategy (Quan et al., 2021). Enterprises with executives who have green experience pay more attention to corporate green technology innovation (Lu Jianci and Jiang Guangsheng, 2022) and may have development strategies different from other enterprises.

To exclude the influence of corporate strategy, (1) referencing the research of Miles et al. (1978) and Liu Hang (2016), enterprises in the same industry are divided into three categories according to different corporate strategy choices: Prospector, Analyzer, and Defender enterprises. Miles et al. believe that Prospector enterprises are enterprises that actively carry out innovation, continuously explore new products and new market opportunities, and face many changes and uncertainties. Defender enterprises are just the opposite. Managers of Defender enterprises position products and markets in a limited range, pursuing corporate stability and reducing uncertainty. The strategic aggressiveness of Analyzer enterprises is exactly between the two. Consistent with the research of Miles et al. (1978) and Liu Hang (2016), this paper constructs a corporate strategy index variable and divides enterprises into three categories. Using corporate strategy choice as a grouping variable, equation(1) is used for regression analysis. The results are shown in Table 10. Columns (1)-(3) correspond to Defender, Analyzer, and Prospector enterprises respectively. The results show that the promotion effect of environmental administrative penalties on the green technology innovation of Defender and Analyzer enterprises is significant. However, the impact on the green technology innovation of Prospector enterprises, which are already committed to innovation strategy, is not significant. This can verify that environmental administrative penalties prompted enterprises that were not originally committed to innovation to carry out green technology innovation, rather than the accidental result of the enterprise's own development strategy, proving the robustness of the baseline regression.

Whether the corporate CEO has green experience also affects the corporate strategy. Referencing the research of Lu Jianci and Jiang Guangsheng (2022), taking whether the CEO has green experience as a grouping variable, equation(1) is used for regression analysis. The results are shown in Table 10. Columns (4) and (5) correspond to CEOs without green experience and with green experience respectively. The results show that for enterprises with CEOs without green experience, the impact of environmental administrative penalties on their green technology innovation is significantly positive. While for the enterprise group with CEOs with green experience, perhaps due to the small sample size, the regression results are not significant. This can exclude the impact of CEO green experience on corporate green technology innovation, proving the robustness of the baseline regression. This paper also used the corporate strategy index and CEO green experience as control variables for regression analysis, which did not affect the baseline regression results. Limited by space, they are not shown for now.

\begin{table}[H]
    \centering
    \caption{Excluding the Impact of Corporate Strategy on Green Technology Innovation}
    \footnotesize
    \begin{tabular}{lccccc}
        \toprule
         & (1) & (2) & (3) & (4) & (5) \\
         & Defender & Analyzer & Prospector & No Green Exp & Has Green Exp \\
        Variable & $Tgreen$ & $Tgreen$ & $Tgreen$ & $Tgreen$ & $Tgreen$ \\
        \midrule
        $fine$ & 0.395** & 0.575* & 0.189 & 0.420*** & 0.398 \\
         & (0.196) & (0.301) & (0.533) & (0.109) & (0.678) \\
        Constant & -11.654*** & -11.179*** & -10.720*** & -11.586*** & -10.082*** \\
         & (1.310) & (1.334) & (1.491) & (0.889) & (2.158) \\
        Controls/FE & Yes & Yes & Yes & Yes & Yes \\
        Observations & 1871 & 1606 & 1001 & 4515 & 472 \\
        $R^2$ & 0.615 & 0.528 & 0.541 & 0.509 & 0.707 \\
        \bottomrule
    \end{tabular}
\end{table}

\section{Mechanism Test of Environmental Administrative Penalties ``Forcing'' Corporate Green Technology Innovation}
Based on the theoretical analysis of this paper, the mechanism by which environmental administrative penalties "force" corporate green technology innovation is the external pressure of media attention and the optimization of internal resource allocation through R\&D investment. Referencing the research of Li Qingyuan and Xiao Zehua (2020), the total volume of media reports is used to measure the external pressure of media attention ($news$). To increase readability, the total volume of media reports is divided by 100. The raw data comes from the CNRDS financial news database. Using the proportion of the number of R\&D personnel to the number of all employees in the company: $R\&D\ rate$, to measure the enterprise's "R\&D investment", the raw data comes from CSMAR. It is worth noting that enterprises carrying out green technology innovation do not necessarily need to increase R\&D investment. Enterprises can increase green technology innovation by squeezing out conventional innovation (Liu Jinke and Xiao Yiyang, 2022). An increase in the proportion of R\&D personnel not only represents the enterprise's commitment to innovation but also highlights the transformation of the enterprise's development strategy (Liu Xin and Xue Youzhi, 2015). Referencing the mechanism test method of Wang Feng and Ge Xing (2022), the following models are constructed for empirical testing.

\begin{equation} \label{eq:3-2}
M_{i,t} = \beta_0 + \beta_1 fine_{i,t} + \beta_n \mathbb{X}_{i,t} + \sum Area + \sum Year + \sum Ind + \varepsilon_{i,t}
\end{equation}

\begin{equation} \label{eq:3-3}
Tgreen_{i,t} = \beta_0 + \beta_1 M_{i,t} + \beta_n \mathbb{X}_{i,t} + \sum Area + \sum Year + \sum Ind + \varepsilon_{i,t}
\end{equation}

Where $M_{i,t}$ represents $news$ or $R\&D\ rate$ of listed company $i$ in year $t$, and other variable settings are consistent with equation(1). First, use equation(2) to test whether environmental administrative penalty intensity ($fine$) can trigger an increase in media attention and the proportion of R\&D personnel of listed companies. Then use equation(3) to test the impact of media attention and R\&D investment on corporate green technology innovation. The mechanism test results are shown in Table 11:

\begin{table}[H]
    \centering
    \caption{Mechanism Test of the Forcing Effect of Environmental Administrative Penalties}
    \footnotesize
    \begin{tabular}{lcccc}
        \toprule
         & \multicolumn{2}{c}{Media Attention} & \multicolumn{2}{c}{R\&D Investment} \\
        Variable & (1) & (2) & (3) & (4) \\
         & $news$ & $Tgreen$ & $R\&D\ rate$ & $Tgreen$ \\
        \midrule
        $fine$ & 0.529** & & 0.016** & \\
         & (0.228) & & (0.007) & \\
        $news$ & & 0.024** & & \\
         & & (0.011) & & \\
        $R\&D\ rate$ & & & & 2.563*** \\
         & & & & (0.400) \\
        Constant & -15.974*** & -10.952*** & -0.042 & -11.483*** \\
         & (2.955) & (0.859) & (0.067) & (0.904) \\
        Controls/FE & Yes & Yes & Yes & Yes \\
        Observations & 4951 & 4951 & 4283 & 4283 \\
        $R^2$ & 0.396 & 0.540 & 0.444 & 0.531 \\
        \bottomrule
    \end{tabular}
\end{table}

The regression results show that the impact of environmental administrative penalty intensity on media attention is positively significant at the 5\% level, and the impact of media attention on corporate green technology innovation is positively significant at the 5\% level. It can be seen that environmental administrative penalties, especially those with higher intensity, can increase the media attention of enterprises. For enterprises receiving larger environmental administrative penalties, both the information on punishment and the information on improvement actions may gain more media attention. Negative media attention can spur enterprises to transform production methods as soon as possible and achieve green innovation. Positive media attention on improvement actions after punishment will also make enterprises further carry out green innovation to maintain a good social image.

The impact of environmental administrative penalty intensity on R\&D investment is positively significant at the 5\% level, and the impact of R\&D investment on corporate green technology innovation is positively significant at the 1\% level. It can be seen that greater environmental administrative penalty intensity makes enterprises realize that the development strategy of long-term active violation is unsustainable, and enterprises invest more resources in R\&D, rather than just increasing green innovation by squeezing out other innovations. Increasing the proportion of R\&D personnel can improve the enterprise's green technology knowledge system, which is conducive to the enterprise carrying out green technology innovation, making the enterprise's technology level comply with environmental protection requirements, and enhancing the competitiveness of corporate products. Thus, hypotheses H2 and H3 are verified.

\section{Comparative Analysis of Environmental Administrative Penalty Intensity and Frequency}
Before 2018, the average fine of environmental administrative penalties in China remained at a relatively low level, while the total number of environmental administrative penalties in China continued to increase. However, since 2018, the average fine has increased significantly, from 49,700 yuan in 2017 to 82,200 yuan in 2018, and the total number of environmental violations by enterprises nationwide began to decrease gradually. The number of environmental administrative penalties received by an enterprise (variable $fine2$ in robustness check 3.3.4) is the research method generally adopted by the current academic community for environmental administrative penalties. The robustness check found that the baseline regression of the number of environmental administrative penalties on corporate green technology innovation was positively significant at the 5\% level. So for corporate managers, are they more concerned about the number of penalties ($fine2$) or the intensity of penalties ($fine$)?

To compare the impact of penalty frequency and penalty intensity on enterprises, the mechanism test method in section 5 is used to explore the mechanism of $fine2$ affecting green technology innovation. The empirical test found that the number of environmental administrative penalties could not trigger more R\&D investment and external media attention. Referencing section 4.5 of the robustness check in this paper, exclude the impact of corporate strategy choice on green technology innovation. The results are shown in Table 12. Columns (3)-(5) correspond to Defender, Analyzer, and Prospector enterprises respectively. After excluding the influence of corporate strategy choice, the impact of the number of environmental administrative penalties ($fine2$) on green technology innovation is no longer significant.

It can be seen that the increase in the number of environmental administrative penalties received by enterprises cannot bring a significant impact on media attention and the enterprise's own R\&D investment. According to the attention economy theory, information is infinite, and attention is limited. Only when an event is important enough is it sufficient to arouse widespread concern and prompt relevant subjects to take countermeasures (Wang Zongsheng and Li Lasheng, 2007). Therefore, when environmental administrative penalties become common, only high-intensity environmental administrative penalties are sufficient to attract media attention and motivate enterprises to increase R\&D investment, ultimately achieving green technology innovation. Therefore, merely increasing the number of environmental administrative penalties is difficult to force Defender and Analyzer enterprises to transform their development methods. Corporate managers are more concerned about environmental administrative penalty intensity, not the number of penalties.

\begin{table}[H]
    \centering
    \caption{Impact of Environmental Administrative Penalty Frequency on R\&D Personnel Ratio and Green Technology Innovation}
    \footnotesize
    \begin{tabular}{lccccc}
        \toprule
         & (1) & (2) & (3) & (4) & (5) \\
         & Media & R\&D Input & Defender & Analyzer & Prospector \\
        Variable & $news$ & $R\&D\ rate$ & $Tgreen$ & $Tgreen$ & $Tgreen$ \\
        \midrule
        $fine2$ & 0.013 & 0.000 & 0.033 & 1.887 & 0.610 \\
         & (0.017) & (0.000) & (0.222) & (1.180) & (1.206) \\
        Constant & -15.776*** & -0.038 & -11.564*** & -11.051*** & -10.681*** \\
         & (3.018) & (0.068) & (1.316) & (1.341) & (1.480) \\
        Controls/FE & Yes & Yes & Yes & Yes & Yes \\
        Observations & 4951 & 4283 & 1905 & 1606 & 1001 \\
        $R^2$ & 0.396 & 0.444 & 0.641 & 0.529 & 0.541 \\
        \bottomrule
    \end{tabular}
\end{table}

\section{Heterogeneity and Innovation Type Analysis}

Considering that heavy pollution industries mainly engage in non-ferrous metal mining and dressing, and manufacturing of chemical raw materials and chemical products, heavy pollution enterprises are more likely to attract the attention of regulatory agencies and are subject to stricter environmental regulations (Guo et al., 2023). Moving from environmental violation to meeting environmental protection requirements, the green technology innovation required for the heavy pollution industry is more difficult and requires more R\&D investment compared to other industries. Therefore, this part divides enterprises into two groups: heavy pollution industries and other industries, for heterogeneity analysis. Columns (1)-(3) are the full sample, non-heavy pollution, and heavy pollution industry samples respectively. From the regression results in Table 13, it can be seen that the full sample, non-heavy pollution, and heavy pollution samples are all significantly positive at least at the 5\% level. There is no difference in direction or significance in the overall green technology innovation impact of environmental administrative penalties on heavy pollution and non-heavy pollution enterprises.

\begin{table}[H]
    \centering
    \caption{Heterogeneity Analysis of the Impact of Environmental Administrative Penalties on Corporate Green Technology Innovation}
    \footnotesize
    \begin{tabular}{lccc}
        \toprule
        Variable & (1) & (2) & (3) \\
         & Full Sample & Non-Heavy Pollution & Heavy Pollution \\
         & $Tgreen$ & $Tgreen$ & $Tgreen$ \\
        \midrule
        $fine$ & 0.422*** & 0.380*** & 0.606** \\
         & (0.103) & (0.104) & (0.260) \\
        Constant & -11.398*** & -11.423*** & -10.684*** \\
         & (0.855) & (1.099) & (1.247) \\
        Controls & Yes & Yes & Yes \\
        Region FE & Yes & Yes & Yes \\
        Time FE & Yes & Yes & Yes \\
        Industry FE & Yes & Yes & Yes \\
        Observations & 5023 & 3201 & 1822 \\
        $R^2$ & 0.536 & 0.555 & 0.513 \\
        \bottomrule
    \end{tabular}
\end{table}

With climate change and environmental problems becoming increasingly serious, green technology innovation is of far-reaching significance to national development. According to the "Patent Law of the People's Republic of China", invention patents possess outstanding substantive features and significant technological progress, and the examination process during authorization is stricter. Invention patents have higher application difficulty and technical value, and are usually used to measure the level of substantive innovation of an enterprise, that is, Substantive Innovation (Guo et al., 2016). Utility model patents refer to practical technical solutions proposed for the shape, structure, or combination of products. The audit conditions are loose, so they are often used by enterprises as an innovation strategy to cater to investors and government supervision, that is, Strategic Innovation. Substantive innovation requires higher R\&D investment and technical level, but brings higher added value and market competitiveness, and can bring substantive improvements to corporate green production and environmental protection; Strategic innovation is relatively easy to implement, but the added value and market competitiveness it brings are limited. Based on this, this paper divides green technology innovation into substantive innovation and strategic innovation, while considering the difference between heavy pollution industries and other industries, to conduct heterogeneity analysis. The research results are shown in Table 14. Columns (1)-(3) correspond to the full sample, non-heavy pollution, and heavy pollution enterprises respectively, and columns (4)-(6) correspond to them.

The study found that for the full sample, environmental administrative penalties can play a significant role in promoting both green substantive innovation and strategic innovation. For the heavy pollution industry, environmental administrative penalties significantly promoted its substantive innovation. For non-heavy pollution enterprises, environmental administrative penalties significantly promoted their strategic innovation. The main business of the heavy pollution industry determines that its production will bring a large amount of pollutant emissions. It is more difficult to improve its environmental friendliness. The technical content of invention patents is higher, which can reduce pollutant emissions more effectively in the long run, make enterprises meet environmental protection standards, and maintain a leading position in the industry. The non-heavy pollution industry itself emits less pollution, and it is easier to achieve environmental compliance compared to the heavy pollution industry, so carrying out strategic innovation is a more cost-effective choice.

The implementation of environmental administrative penalties guides different types of enterprises to carry out strategies that are consistent with long-term profit maximization and China's sustainable development direction. It can be seen that the high efficiency and directionality of environmental administrative penalties as a government environmental governance method are significant. This result has reference significance for China to achieve green development.

\begin{table}[H]
    \centering
    \caption{Analysis of Corporate Green Technology Innovation Types and Corporate Heterogeneity}
    \footnotesize
    \begin{tabular}{lcccccc}
        \toprule
        Variable & (1) & (2) & (3) & (4) & (5) & (6) \\
         & Green Invention & Green Invention & Green Invention & Green Utility & Green Utility & Green Utility \\
         & Full Sample & Non-Heavy & Heavy & Full Sample & Non-Heavy & Heavy \\
        \midrule
        $fine$ & 0.302** & 0.178 & 0.743*** & 0.294*** & 0.342** & 0.158 \\
         & (0.125) & (0.128) & (0.239) & (0.107) & (0.144) & (0.213) \\
        Constant & -10.161*** & -10.593*** & -9.338*** & -8.775*** & -8.839*** & -7.714*** \\
         & (0.828) & (1.057) & (1.098) & (0.704) & (0.934) & (1.088) \\
        Controls & Yes & Yes & Yes & Yes & Yes & Yes \\
        Region FE & Yes & Yes & Yes & Yes & Yes & Yes \\
        Time FE & Yes & Yes & Yes & Yes & Yes & Yes \\
        Industry FE & Yes & Yes & Yes & Yes & Yes & Yes \\
        Observations & 5023 & 3201 & 1822 & 5023 & 3201 & 1822 \\
        $R^2$ & 0.504 & 0.523 & 0.488 & 0.506 & 0.528 & 0.475 \\
        \bottomrule
    \end{tabular}
\end{table}

\section{Extension Analysis: Peer Effect and Moderating Effect}

\subsection{Peer Effect Analysis}
From the theoretical analysis section, it is known that when the government implements environmental administrative penalties, it transmits a signal of environmental administrative law enforcement intensity to local enterprises. Peer enterprises make expectations of illegal costs and illegal benefits based on observation. Do environmental administrative penalties have a peer effect? Is there a difference in the peer effect between penalty frequency and penalty intensity? These questions need to be verified, referencing the research of Shi Guifeng (2015). Construct variables as shown in Table 15.

\begin{table}[H]
    \centering
    \caption{Definition of Variables Measuring Peer Effects}
    \small
    \begin{tabular}{ll}
        \toprule
        \textbf{Variable} & \textbf{Meaning} \\
        \midrule
        $MeanArea_{j,t}$ & Average number of penalties for enterprises in region $j$ in year $t$ \\
        $StrmeanArea_{j,t}$ & Average penalty intensity for enterprises in region $j$ in year $t$ \\
        $MeanAreaInd_{j,k,t}$ & Average number of penalties for enterprises in industry $k$, region $j$, year $t$ \\
        $StrmeanAreaInd_{j,k,t}$ & Average penalty intensity for enterprises in industry $k$, region $j$, year $t$ \\
        \bottomrule
    \end{tabular}
\end{table}

Considering that the transmission of environmental administrative penalty information has a certain time lag, lag the above variables by one period, replace the $fine$ variable in equation(1), and perform regression. To exclude the impact of current environmental administrative penalties on corporate green technology innovation, regression is performed only on samples that were not penalized in the current period. The regression results are shown in Table 16. The "peer effect" of the average number of penalties in the region is not significant; the "peer effect" is significantly positive after adding industry conditions; the "peer effect" of penalty intensity is significantly positive, and hypothesis H4 is verified. When the government imposes environmental administrative penalties on violating enterprises, it releases a signal of environmental administrative law enforcement intensity to the outside world. Peer enterprises re-weigh the relationship between costs and future benefits, carry out green technology innovation, and avoid violations. At the same time, compared to the average number of penalties, corporate managers are more concerned about the average penalty intensity. Compared to the penalty situation in the same region, corporate managers are more concerned about the penalty situation in the same region and the same industry.

Combining the analysis of "media attention", "R\&D investment" and "peer effect", the phenomenon of "more penalties, more violations" in environmental administrative penalties in China from 2013 to 2017 can be understood. The number of penalties has little impact on corporate operating costs and future expectations, and it is difficult to incentivize enterprises to increase R\&D investment, and the level of corporate innovation has not been substantially improved. Moreover, the number of penalties is difficult to deter peer enterprises from carrying out green technology innovation, so environmental administrative penalties are "increasing". This research result has reference significance for deepening local governments' understanding of environmental protection law enforcement, reducing government "local protectionism" behavior, and achieving a win-win situation for environmental protection and economic development.

\begin{table}[H]
    \centering
    \caption{Peer Effect Test}
    \footnotesize
    \begin{tabular}{lcccc}
        \toprule
        Variable & (1) & (2) & (3) & (4) \\
         & $Tgreen$ & $Tgreen$ & $Tgreen$ & $Tgreen$ \\
        \midrule
        $L.MeanArea_{j,t}$ & -0.007 & & & \\
         & (0.014) & & & \\
        $L.StrmeanArea_{j,t}$ & & 1.856** & & \\
         & & (0.855) & & \\
        $L.MeanAreaInd_{j,k,t}$ & & & 0.011** & \\
         & & & (0.006) & \\
        $L.StrmeanAreaInd_{j,k,t}$ & & & & 0.374*** \\
         & & & & (0.078) \\
        Constant & -10.826*** & -10.825*** & -10.837*** & -10.885*** \\
         & (0.933) & (0.935) & (0.935) & (0.933) \\
        Controls/FE & Yes & Yes & Yes & Yes \\
        Observations & 2853 & 2853 & 2853 & 2853 \\
        $R^2$ & 0.502 & 0.503 & 0.503 & 0.503 \\
        \bottomrule
    \end{tabular}
\end{table}

\subsection{Moderating Effect of Government Environmental Subsidies}
The government is an important designer and participant in the national innovation system. It not only imposes environmental administrative penalties on enterprises but also provides environmental subsidies. Whether the impact of government subsidies on corporate innovation is "stimulating" or "weakening" has been debated (Guo Yue, 2018). The "stimulating" view believes that government subsidies can make up for market failures in the innovation process and promote corporate R\&D resource investment (Jaffe et al., 2015). The "weakening" view believes that information asymmetry between government and enterprises will cause subsidies to have a "reverse" guiding role, causing over-investment (Wei Zhihua et al., 2015), and the direct support object of government environmental subsidies is the enterprise's environmental protection engineering investment, not targeted at green technology innovation. Corporate catering to the government and opportunism may "weaken" corporate green technology innovation (Li Qingyuan and Xiao Zehua, 2020).

So, how will government environmental subsidies affect corporate responses to environmental administrative penalties? This part references the research of Guo Yue (2018). The raw data on government environmental subsidies comes from CSMAR. Text matching is performed on the "Government Subsidy Details" under the "Non-operating Income" subject in the notes to the company's annual financial statements, screening for keywords related to environmental protection to obtain the total amount of corporate environmental subsidies. To make the impact of government environmental subsidies on enterprises of different scales comparable, the environmental subsidy amount is divided by the total assets of the enterprise to obtain the government environmental subsidy intensity. Construct a dummy variable $subsidy$. If the government environmental subsidy intensity is greater than the sample median, then $subsidy=1$, otherwise it is 0. Add $subsidy$ and the interaction term of $subsidy$ and $fine$ to equation(1) for regression verification. There are 653 observations where $fine \times subsidy > 0$, accounting for about 1/3 of observations where $fine > 0$. Considering that heavy pollution industries mainly have main businesses such as non-ferrous metal mining and dressing, chemical raw materials and chemical products manufacturing, heavy pollution enterprises are more likely to attract the attention of regulatory agencies and be subject to stricter environmental regulations (Guo et al., 2023). The green technology innovation required for the heavy pollution industry to move from environmental violation to compliance with environmental protection requirements is more difficult and requires more R\&D investment than other industries. Therefore, this part divides enterprises into two groups: heavy pollution industry and other industries, for heterogeneity analysis. Columns (1)-(3) are the full sample, non-heavy pollution, and heavy pollution industry samples respectively.

The heterogeneity analysis in Section 7 shows that environmental administrative penalties have a promotion effect on green technology innovation in both non-heavy pollution and heavy pollution enterprises. Combining the regression results in Table 17, overall, government subsidies show a negative moderating effect, which is significant at the 5\% level. The negative moderating effect of non-heavy pollution enterprises is particularly obvious, and the moderating effect of heavy pollution is not significant. Environmental subsidies weakened the promotion effect of environmental administrative penalties on green technology innovation in non-heavy pollution enterprises. Government environmental subsidies target direct investment in environmental protection projects. If enterprises can meet environmental protection requirements by carrying out environmental protection engineering investment, there is no need to carry out green technology innovation. For the purpose of catering to the government, reducing risks, and reducing costs, corporate green technology innovation decreases. The environmental impact of the main business of the non-heavy pollution industry is smaller than that of the heavy pollution industry, and the difficulty of rectification when its involved business violates environmental protection regulations is smaller. Therefore, the negative moderating impact coefficient of government environmental subsidies on non-heavy pollution industries is larger. The main business of the heavy pollution industry is highly polluting, and it is usually difficult to make the enterprise comply with environmental protection requirements for a long time through simple environmental protection engineering construction. Carrying out green technology innovation has become a necessary option.

\begin{table}[H]
    \centering
    \caption{Moderating Effect of Government Environmental Subsidies}
    \footnotesize
    \begin{tabular}{lccc}
        \toprule
        Variable & (1) & (2) & (3) \\
         & Full Sample & Non-Heavy Pollution & Heavy Pollution \\
         & $Tgreen$ & $Tgreen$ & $Tgreen$ \\
        \midrule
        $fine$ & 0.865*** & 1.184*** & 0.552 \\
         & (0.258) & (0.372) & (0.370) \\
        $subsidy$ & 0.180*** & 0.245*** & 0.059 \\
         & (0.046) & (0.061) & (0.065) \\
        $fine \times subsidy$ & -0.550** & -0.914** & 0.099 \\
         & (0.269) & (0.374) & (0.497) \\
        Constant & -11.797*** & -11.993*** & -10.786*** \\
         & (0.881) & (1.125) & (1.269) \\
        Controls/FE & Yes & Yes & Yes \\
        Observations & 5023 & 3201 & 1822 \\
        $R^2$ & 0.539 & 0.560 & 0.514 \\
        \bottomrule
    \end{tabular}
\end{table}

\section{Conclusions and Policy Recommendations}

Environmental administrative penalties constitute a crucial component of the rule of law in China's environmental protection. They assist regulatory agencies in effectively enforcing environmental laws and regulations, promoting compliance among individuals and enterprises, and achieving green development. Since 2018, the average value of environmental administrative fines for Chinese enterprises has increased significantly; however, the impact on enterprises remains unclear. Based on this problem orientation, this paper investigates the impact of increased environmental administrative penalty intensity on corporate green technology innovation.

\subsection{Research Conclusions}

This paper conducts empirical research based on data from A-share listed companies from 2015 to 2020. The study finds that an increase in the intensity of environmental administrative penalties can ``force'' (reverse transmission effect) enterprises to engage in green technology innovation. This conclusion remains robust after excluding endogeneity issues, employing Propensity Score Matching (PSM), excluding competitive policies, replacing variable measurement methods, and controlling for corporate strategic choices. This indicates that when enterprises face high-intensity environmental administrative penalties, the pressure of ``direct costs'' combined with external pressures from the legal system and public opinion makes them realize the unsustainability of illegal behaviors. Consequently, enterprises engage in green technology innovation to maintain a positive social image and achieve long-term profit maximization.

Mechanism tests reveal that environmental administrative penalties promote corporate green technology innovation through external pressure from media attention and the optimization of internal resource allocation via increased R\&D investment. High-intensity environmental administrative penalties can increase media attention on enterprises. Greater media scrutiny forces managers to weigh the consequences of environmental pollution, prompting enterprises to carry out green technology innovation. Furthermore, environmental administrative penalties create a demand for environmental protection and green transformation, driving enterprises to increase R\&D investment and improve their green technology knowledge systems, which is conducive to green technology innovation.

A comparative analysis of penalty frequency versus penalty intensity reveals that penalty frequency does not increase ``media attention'' or ``R\&D investment.'' After excluding the influence of corporate strategic choices, the impact of penalty frequency on corporate green technology innovation is no longer significant. It can be inferred that the green technology innovation effect of environmental administrative penalties mainly stems from the enhancement of penalty intensity rather than the increase in penalty frequency; corporate managers pay more attention to the intensity of penalties than the number of times they are penalized. Therefore, the total number of environmental violation incidents decreased only after China increased the intensity of environmental administrative penalties in 2018.

Environmental administrative penalties also exhibit a ``peer effect,'' promoting green technology innovation in non-penalized enterprises within the same industry and region. However, government environmental subsidies have a weakening effect on the ``forcing effect'' of environmental administrative penalties, which is particularly evident for non-heavy polluting enterprises. If enterprises can meet environmental protection requirements merely by investing in environmental engineering projects without the need for further green technology innovation, they may reduce their green technology innovation efforts for the purposes of catering to the government, reducing risks, and lowering costs.

\subsection{Policy Recommendations}

Based on the research findings of this paper, the following policy recommendations are proposed:

\textbf{First, improve the environmental supervision mechanism, enforce laws strictly, and appropriately increase the intensity of environmental administrative penalties.} The improvement of the environmental supervision mechanism and the symmetry of environmental pollution information are prerequisites for achieving judicial justice and the basis for assessing the negative externalities of corporate pollution. These directly affect the fairness of corporate operations and the scientific nature of government environmental governance. The central government should establish a sound environmental supervision mechanism to reduce the impact of local protectionism on the authenticity of environmental pollution information. Through empirical evidence, this paper verifies the promoting effect of environmental administrative penalty intensity on corporate green technology innovation. Local governments should improve their cognitive levels, eliminate short-sighted behaviors such as local protectionism, and enforce laws strictly. Enterprises of different sizes react differently to fines of the same amount, and penalty intensity is of greater concern to corporate managers than penalty frequency. When environmental authorities impose administrative penalties, they should consider corporate heterogeneity to ensure that the cost of violation exceeds the gains from violation, thereby eliminating ``active violation'' behaviors by enterprises.

\textbf{Second, focus on the long-term development of enterprises and establish a ``Look-back'' mechanism.} Empirical tests find that the external pressure brought by media attention has a significant promoting effect on corporate green technology innovation. Guiding enterprises to focus on maximizing long-term profits is an important guarantee for achieving a win-win situation for corporate long-term development and environmental protection. Local environmental authorities can learn from the mechanisms of media scrutiny and the Central Environmental Inspection's ``Look-back'' mechanism. By establishing a local environmental administrative penalty ``Look-back'' mechanism, authorities can monitor the long-term environmental performance of violating enterprises and promote a tangible improvement in their green development levels.

\textbf{Third, improve information disclosure and establish a specialized ``Peer Enterprise'' warning mechanism to achieve deterrence.} Research results indicate that when the government implements environmental administrative penalties, it signals the intensity of environmental administrative law enforcement to local enterprises. Peer enterprises, based on observation, form expectations regarding the costs and benefits of violations and engage in green technology innovation when the intensity of penalties is high. Therefore, corporate decision-making behavior is influenced by administrative law enforcement information. Administrative law enforcement departments should improve information disclosure mechanisms, clarifying the scope, content, methods, and procedures of environmental administrative penalty information disclosure. This ensures that penalty information is released to the public comprehensively, timely, and accurately. Furthermore, a specialized warning mechanism for ``peer enterprises'' should be established to urge enterprises to consciously abide by environmental protection laws and regulations, strengthen environmental management, and achieve the effect of warning others against following bad examples.

\textbf{Fourth, pay attention to the interaction of policies.} In the research sample of this paper, approximately one-third of the penalized enterprises received environmental subsidies greater than the median, and environmental subsidies were found to weaken the innovation effect of environmental administrative penalties. Therefore, when the government conducts reviews for enterprise assistance policies such as environmental subsidies, it should fully examine the past performance of enterprises. For enterprises that have previously received environmental administrative penalties, the government should track the usage of their environmental subsidies more closely and supervise them to address both the symptoms and root causes of pollution. In the process of administrative governance, the government should fully consider the interaction of policies to avoid the phenomenon of policy effects canceling each other out, ensuring that policies work in synergy to promote the long-term sustainable development of enterprises.

\subsection{Research Outlook}

Through theoretical analysis and empirical testing, this paper has explored the impact of environmental administrative penalty intensity on corporate green technology innovation and its mechanisms, providing theoretical references and empirical support for improving China's environmental rule of law and the modernization of governance capabilities. However, this paper still has limitations, and future research can be expanded in the following aspects:

\textbf{First, examine the implementation effects of environmental administrative penalties over a longer time dimension.} constrained by the availability of data, this paper only considered environmental administrative penalty data from 2015 to 2020. Data covering a longer period would allow for further tracking of the status of environmental administrative penalties in China, enhancing the timeliness and robustness of the research conclusions. Meanwhile, China began implementing the \textit{Measures for Ecological and Environmental Administrative Punishment} on July 1, 2023. Research based on more current environmental rule of law may yield more comprehensive conclusions. Additionally, employing more complete instrumental variables and mechanism testing methods can further improve the robustness of the research conclusions.

\textbf{Second, further explore the most appropriate range of environmental administrative penalty intensity.} This paper found that the intensity of environmental administrative penalties has caused a significant increase in the operating costs of state-owned enterprises (SOEs), but the increase in operating costs for non-SOEs was not significant (not reported due to space limitations). It can be inferred that there exists an optimal range for the intensity of environmental administrative penalties. It is possible that for most enterprises currently, the intensity of penalties has not yet reached a level that inhibits corporate growth, and the sample of enterprises subject to high-intensity penalties is small; thus, threshold effects and non-linear regressions were not significant. As the information disclosure system improves and the intensity of environmental administrative penalties increases in the future, the number of enterprises subject to high-intensity penalties will increase, allowing for a deeper exploration of the optimal environmental administrative penalty intensity.

\newpage
\bibliographystyle{plainnat}
\bibliography{references}
\end{document}